\documentclass[a4paper,11pt]{scrartcl}
\usepackage[utf8]{inputenc}
\usepackage[spanish]{babel}
\usepackage{url}
\usepackage{hyperref}
\usepackage{biblatex}
\addbibresource{references.bib}

%opening
\title{Realidad Aumentada}
\subtitle{Trabajo teórico de la asignatura Interacción Persona-Ordenador 1}
\author{Piotr Maliszewski\\Maciej Nalepa}
\setlength{\parindent}{5ex}


\begin{document}

\maketitle

\begin{abstract}

Esta actividad consiste en la realización de un trabajo teórico sobre interacción avanzada y
nuevos paradigmas de interacción. Para ello el alumno deberá investigar sobre las nuevas
formas de Interacción Persona-Ordenador que están apareciendo en los últimos años, y elaborar un
informe sobre dicha temática, indicando un posible dominio de aplicación, proponiendo y
describiendo un posible escenario de uso.

\end{abstract}

% Introducción y definición de conceptos de la temática seleccionada
\section{Introducción y definición de conceptos}
\subsection{¿Qué es Realidad Aumentada?}
La realidad aumentada es una tecnología que permite combinar los elementos virtuales con el entorno real y representarlos a tiempo real. Gracias a esta tecnología podemos aportar conocimiento al entorno que nos rodea. Para poder entender bien lo que realmente significa realidad aumentada hay que diferenciar desde un principio lo que es la realidad así como la realidad virtual. La definición de la realidad parece más simple de lo que es. Se define el término realidad como todo aquello que existe en el mundo real; mientras que la realidad es aquel entorno que se genera de manera digital y que nos da la sensación de estar inmerso en él.


\subsection{Realidad Virtual y Mixta}
A menudo se puede oír varias nociones como: realidad virtual, realidad aumentada y incluso realidad mixta. Sin embargo esto no es lo mismo. Primero de nada, la realidad virtual construye un mundo nuevo en el que nos sumergimos, mientras que, en la realidad aumentada objectos virtuales parecen aparecer en nuestro propio entorno. En otras palabras la realidad virtual crea un entorno cual podemos percibir en varias maneras y nosotros, es decir nuestros cuerpos no somos la parte coherente del sistema creado. La realidad aumentada aparece alrededor de nosotros. No percibimos todo el entorno artificial, sino dejanos experimentar objectos generados en nuestra vida.
\par La realidad mixta es la más nueva tecnología de las dichas y intenta fusionar las ventajas de sus precedesoras. La cosa que difencia la realidad mixta de la realidad aumentada es que los elementos virtuales interactúan directamente con el entorno. No solo aparecen alrededor, pero también pueden detectar obstáculos y ajustar sus comportamientos a esos. Una grande parte de esta tecnología depende de la realidad aumentada porque está basada en ella. [?...] 

% Estado actual del tema y algunos desarrollos o aplicaciones de interés
\section{Estado actual del tema}
\subsection{Rango(?) de uso actual}
Mucha gente oyó por primera vez de esta tecnología gracias al impacto del juego Pókemon Go \cite{pokemongo}. Los juegos son un gran mercado y aunque alguien no le gusta la expansión de este mercado, no se puede negar que el mercado de juegos nos permite desarrollar esto. Sin embargo esta tecnología no nos deja solo diversitar nuestro ocio, pero también desarollar el deporte, la medicina, el marketing, la ingenería, la educación y muchas otras cosas. Cada año más empresas utilizan la realidad aumentada para hacer el trabajo más eficaz, cómodo y rentable. Afortunadamente no solo los empleados pueden alegrar de esta tecnología. También la ordinaria gente
\subsection{Capacitar a los trabajadores}
La empresa Lowe's es un buen example del uso actual de esta tecnología. Lowe's es una compañia minorista donde el servicio al cliente es muy vigente. Hablar con una persona real es la parte de nuestra naturaleza humana. Por esto es importante que la capacitación de agentes de servicio sea sólida. Personal de servicio cual vende una amplia gama de productos debe ser capaz de explicar todo que se ofrece. Para que el entrenamiento sea más rápido, la empresa usa la realidad aumentada para hablitar a los trabajadores el acceso a los productos sin la necesidad de manejarlos físicamente. La compañia no solo usa esta tecnología para capacitar su personel, sino que ofrece a los clientes los modelos de alta fidelidad para ver y experimentar las características de los productos antes comprarlos.
\subsection{Reducción de los incertidumbres}
Si  una gran empresa de compras en línea, es posible que tenga un problema. A pesar de lo que comprar en línea benificia mucho a la gente, también tiene sus desventajas. No se puede examinar físicamente el producto y por esto más gente devuelve productos comprados en línea que los comprados en las tiendas físicas. Ikea decidió resolver este problema introduciendo una aplicación moderna cual permite disminuir los incertidumbres entre los clientes. Dicha aplicación da la impresión completa permitiendo a los clientes probar los muebles en sus hogares. Esta solución no solo deja a Ikea reducir los costes, sino también ahorra el estrés e las molestias antes de unas compra.

\subsection{envases interactivas}
Un otro interesante ejemplo de la penetración de la realidad aumentada en nuestra cotidianidad es lo que realiza la compañia Heinz con sus publicidades. Los clientes pueden orientar sus teléfonos hasta el producto y ver las receptas. La vista de la comida puede estimular a la compra. La oferta se muestra como un libro en el producto. Esa posibilidad fue introducido ya en 2011 año.

\subsection{Software}
\subsection{Hardware}
% Dominio de aplicación elegido y descripción de un posible escenario de uso (se podrán describir los requisitos hardware y software, arquitectura, ...)
\section{Dominio de aplicación y posible escenario de uso}

\subsection{Juegos}
\subsection{Tourismo}
\subsection{Diseño}

% Conclusiones y opinión personal
\section{Conclusiones}

% Referencias bibliográficas consultadas
\section{Referencias}
\printbibliography

https://www.neosentec.com/realidad-aumentada/


to do
Codigo QR:
https://www.redalyc.org/pdf/1995/199524426028.pdf
uso:
http://www.centrocp.com/realidad-aumentada-y-virtual-en-el-marco-de-la-discapacidad-e-inclusion-desde-una-perspectiva-universitaria/
medicina:
https://www.xataka.com/wearables/una-gafas-de-realidad-aumentada-ayudan-a-diferenciar-las-celulas-cancerigenas-en-tiempo-real
Mixta:
https://www.espaciobim.com/realidad-mixta
varios bloges:
https://www.espaciobim.com/

uso en empresas:
https://www.tworeality.com/las-grandes-empresas-ya-utilizan-realidad-virtual-en-la-formacion-de-sus-empleados/

uso en ikea por clientes:
https://blog.hubspot.es/service/ejemplos-realidad-aumentada

lowe's:
https://www.lowesinnovationlabs.com/lowes3d

uso en ikea:
https://www.youtube.com/watch?v=ALpVlVsH6M8

comprar en línea vs en tiendas
https://www.shopify.com/enterprise/ecommerce-returns
\end{document}
