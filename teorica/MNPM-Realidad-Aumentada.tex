\documentclass[a4paper,11pt]{scrartcl}
\usepackage[utf8]{inputenc}
\usepackage[spanish]{babel}

%opening
\title{Realidad Aumentada}
\subtitle{Trabajo teórico de la asignatura Interacción Persona-Ordenador 1}
\author{Piotr Maliszewski\\Maciej Nalepa}

\begin{document}

\maketitle

\begin{abstract}

Esta actividad consiste en la realización de un trabajo teórico sobre interacción avanzada y
nuevos paradigmas de interacción. Para ello el alumno deberá investigar sobre las nuevas
formas de Interacción Persona-Ordenador que están apareciendo en los últimos años, y elaborar un
informe sobre dicha temática, indicando un posible dominio de aplicación, proponiendo y
describiendo un posible escenario de uso.

\end{abstract}

% Introducción y definición de conceptos de la temática seleccionada
\section{Introducción y definición de conceptos}
\subsection{Qué es Realidad Aumentada}
\subsection{Realidad Virtual, Aumentada y Mixed}

% Estado actual del tema y algunos desarrollos o aplicaciones de interés
\section{Estado actual del tema}

% Dominio de aplicación elegido y descripción de un posible escenario de uso (se podrán describir los requisitos hardware y software, arquitectura, ...)
\section{Dominio de aplicación y posible escenario de uso}
\subsection{Juegos}
\subsection{Tourismo}
\subsection{Diseño}

% Conclusiones y opinión personal
\section{Conclusiones}

% Referencias bibliográficas consultadas
\section{Referencias}

\end{document}
