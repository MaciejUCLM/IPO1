\documentclass[a4paper,11pt]{scrartcl}
\usepackage[utf8]{inputenc}
\usepackage[spanish]{babel}

%opening
\title{Realidad Aumentada}
\subtitle{Trabajo teórico de la asignatura Interacción Persona-Ordenador 1}
\author{Piotr Maliszewski\\Maciej Nalepa}

\begin{document}

\maketitle

\begin{abstract}

Esta actividad consiste en la realización de un trabajo teórico sobre interacción avanzada y
nuevos paradigmas de interacción. Para ello el alumno deberá investigar sobre las nuevas
formas de Interacción Persona-Ordenador que están apareciendo en los últimos años, y elaborar un
informe sobre dicha temática, indicando un posible dominio de aplicación, proponiendo y
describiendo un posible escenario de uso.

\end{abstract}

% Introducción y definición de conceptos de la temática seleccionada
\section{Introducción y definición de conceptos}
\subsection{¿Qué es Realidad Aumentada?}
La realidad aumentada es una tecnología que permite combinar los elementos virtuales con el entorno real y representarlos a tiempo real. Gracias a esta tecnología podemos aportar conocimiento al entorno que nos rodea. Para poder entender bien lo que realmente significa realidad aumentada hay que diferenciar desde un principio lo que es la realidad así como la realidad virtual. La definición de la realidad parece más simple de lo que es. Se define el término realidad como todo aquello que existe en el mundo real; mientras que la realidad es aquel entorno que se genera de manera digital y que nos da la sensación de estar inmerso en él.


\subsection{Realidad Virtual y Mixta}
A menudo se puede oír varias nociones como: realidad virtual, realidad aumentada y inluso realidad mixta. Sin embargo esto no es lo mismo. Primero de nada, la realidad virtual construye un mundo nuevo en el que nos sumergimos, mientras que, en la realidad aumentada objectos virtuales parecen aparecer en nuestra propio entorno. Para ilustrarlo mejor,

% Estado actual del tema y algunos desarrollos o aplicaciones de interés
\section{Estado actual del tema}

% Dominio de aplicación elegido y descripción de un posible escenario de uso (se podrán describir los requisitos hardware y software, arquitectura, ...)
\section{Dominio de aplicación y posible escenario de uso}

\subsection{Juegos}
\subsection{Tourismo}
\subsection{Diseño}

% Conclusiones y opinión personal
\section{Conclusiones}

% Referencias bibliográficas consultadas
\section{Referencias}
%\bibliographystyle{plain}
%\bibliography{references}
https://www.neosentec.com/realidad-aumentada/

to do
Codigo QR:
https://www.redalyc.org/pdf/1995/199524426028.pdf
uso:
http://www.centrocp.com/realidad-aumentada-y-virtual-en-el-marco-de-la-discapacidad-e-inclusion-desde-una-perspectiva-universitaria/
medicina:
https://www.xataka.com/wearables/una-gafas-de-realidad-aumentada-ayudan-a-diferenciar-las-celulas-cancerigenas-en-tiempo-real
\end{document}
